\chapter{Theory}
\label{ch:theory}
In order to interpret any experimental result, it is of paramount importance to understand
the underlying model governing the physical processes in question. Modern physics knows a
large number of rather successful theories all dedicated to describing different mass and 
energy scales. An example is the theory of classical mechanics, which manages to describe the 
physics of `daily life' very well. However, it breaks down when velocities approach
the speed of light and has to be incorporated into a broader theory, namely that of relativity.

This specific example already suggests that different physical theories are valid only in a 
certain energy range and describe only a certain `type'\footnote{In this particular example
electromagnetic interactions are -- for instance -- not described at all.} of physical process. 
This fact is also true for the case of particle physics. The relevant theory is called the 
\textit{`Standard Model'} and will be described hereafter. Further into the chapter,
a short description of the pitfalls of the standard model will be given with some explanation
on possible solutions.

\section{The Standard Model}
\label{sec:standardmodel}
The Standard Model (SM) of particle physics provides the theoretical framework that
describes all fundamental particles and the forces that act between them, with the one
exception of gravity. Despite a few drawbacks that will be described later (see Section~\ref{sub:sm_shorts})
it has been an overwhelmingly successful theory, capable of describing experimental data
with a precision that is simply outstanding.

\subsection{Particle content of the Standard Model}
\label{sub:sm_particles}

\subsection{Interactions}
\label{sub:sm_interactions}

\subsection{Shortcomings of the Standard Model}
\label{sub:sm_shorts}

\section{Supersymmetry}
\label{sec:susy}

\subsection{Particle content}
\label{sub:susy_particles}

\subsection{Observables for searches for Supersymmetry}
\label{sub:susy_observables}

\section{Remaining open questions}
\label{sec:theory_remains}
