\chapter{Theory}
\label{ch:theory}
In order to interpret any experimental result, it is of paramount importance to understand
the underlying model governing the physical processes in question. Modern physics knows a
large number of rather successful theories all dedicated to describing different mass and 
energy scales. An example is the theory of classical mechanics, which manages to describe the 
physics of `daily life' very well. However, it breaks down when velocities approach
the speed of light and has to be incorporated into a broader theory, namely that of relativity.

This specific example already suggests that different physical theories are valid only in a 
certain energy range and describe only a certain `type'\footnote{In this particular example
electromagnetic interactions are -- for instance -- not described at all.} of physical process. 
This fact is also true for the case of particle physics. The relevant theory is called the 
\textit{`Standard Model'} and will be described hereafter. Further into the chapter,
a short description of the pitfalls of the standard model will be given with some explanation
on possible solutions, leading to the introduction of Supersymmetry.

\section{The Standard Model}
\label{sec:standardmodel}
The Standard Model (SM) of particle physics provides the theoretical framework that
describes all fundamental particles and the forces that act between them, with the one
exception of gravity. The term `fundamental' (or `elementary) describes the characteristic of a particle to
not consist of smaller parts. Despite a few drawbacks that will be described later (see Section~\ref{sub:sm_shorts})
the SM has been an overwhelmingly successful theory, capable of describing experimental data
with a precision that is simply outstanding among physical theories. In addition to the 
precision of the theory, it also numerous very fundamental and bold predictions in the past,
of which a large number have been confirmed by experiments.

\subsection{Particle content of the Standard Model}
\label{sub:sm_particles}

The Standard Model knows two types of fundamental particles. On the one hand are the particles that comprise
all known matter, so-called \textit{fermions}, while on the other hand the interaction between the fermions
are mediated by so-called \textit{bosons}. 

\subsubsection*{Fermions}

The main distinguishing property of fermions is the absolute value of their internal angular momentum -- spin -- of \num{1/2}. 
The massive fermions are divided into two families, so called \textit{leptons} and
\textit{quarks}. While the behavior of leptons is governed by the electromagnetic and the weak force, the quark
sector is governed by all forces in the SM, including the strong force. Both families of fermions come in three different
generations, which are ordered by their mass but carry otherwise the same properties in terms of charges. Only members of the lightest of these generations are stable, and as a consequence all known 
matter is comprised of only three fundamental particles\footnote{This is only true in the limit of vanishing neutrino masses, which is a good 
enough approximation for all intents and purposes. Additionally, massive neutrinos pose an intrinsic problem to the SM itself.}.
A list and classification of all fermions can be found in Table~\ref{tab:fermions}.


\begin{table}
\bgroup
\def\arraystretch{1.2}
    \small
    \centering
    \caption{Elementary fermions of the Standard model with their masses and charges. $Q$ denotes the electric charge, $T^3$ the weak isospin, which can be regarded
    as the charge of the weak force. $Y_W = 2\cdot(Q - T_3)$ is the weak hypercharge and combines electric and weak charge. Particles carrying strong charge $Y_S$ are subject to the strong force.}
    \label{tab:fermions}
    \begin{tabular}{ l l l l c c c c  }
                                      & 1st generation                              & 2nd generation                            & 3rd generation                            & $Q$                     & $T_3$          & $Y_W$          & $Y_S$ \\ \hline \hline
    \multirow{5}{*}{\textbf{Leptons}} & \textbf{electron} \footnotesize{(0.5 \mev)} & \textbf{muon} \footnotesize{(106 \mev)}   & \textbf{tau} \footnotesize{(1.8 \gev)}    & \multicolumn{4}{c}{} \\
                                      & $e_L$                                       & $\mu_L$                                   & $\tau_L$                                  & -1                      & $\frac{1}{2}$  & -1             & \footnotesize{no} \\
                                      & $e_R$                                       & $\mu_R$                                   & $\tau_R$                                  & -1                      & 0              & -2             & \footnotesize{no} \\ \cline{2-8}
                                      & \textbf{$e$-neutrino}                       & \textbf{$\mu$-neutrino}                   & \textbf{$\tau$-neutrino}                  & \multicolumn{4}{c}{} \\
                                      & $\nu_e$                                     & $\nu_\mu$                                 & $\nu_\tau$                                & 0                       & -$\frac{1}{2}$ & -1             & \footnotesize{no} \\ \hline \hline
    \multirow{6}{*}{\textbf{Quarks}}  & \textbf{up} \footnotesize{(2.3 \mev)}       & \textbf{charm} \footnotesize{(1.29 \gev)} & \textbf{top} \footnotesize{(173 \gev)}    & \multicolumn{4}{c}{} \\
                                      & $u_L$                                       & $c_L$                                     & $t_L$                                     & $+\frac{2}{3}$          & $\frac{1}{2}$  & $\frac{1}{3}$  & \footnotesize{yes} \\
                                      & $u_R$                                       & $c_R$                                     & $t_R$                                     & $+\frac{2}{3}$          & 0              & $\frac{4}{3}$  & \footnotesize{yes} \\ \cline{2-8}
                                      & \textbf{down} \footnotesize{(4.8 \mev)}     & \textbf{strange} \footnotesize{(95 \mev)} & \textbf{bottom} \footnotesize{(4.2 \gev)} & \multicolumn{4}{c}{} \\
                                      & $d_L$                                       & $s_L$                                     & $b_L$                                     & $-\frac{1}{3}$          & $-\frac{1}{2}$ & $\frac{1}{3}$  & \footnotesize{yes} \\
                                      & $d_R$                                       & $s_R$                                     & $b_R$                                     & $-\frac{1}{3}$          & 0              & $-\frac{2}{3}$ & \footnotesize{yes} \\
    \hline\hline
    \end{tabular}
\egroup
\end{table}


\subsection{Interactions}
\label{sub:sm_interactions}

explain a bit the SM lagrangian

\subsection{Shortcomings of the Standard Model}
\label{sub:sm_shorts}

dark matter, dark energy, hierarchy problem, gravity perhaps


\section{Supersymmetry}
\label{sec:susy}

explain the general idea of susy. essentially reducing the primer to 3 pages

\subsection{Particle content}
\label{sub:susy_particles}

talk a bit about the different particles, why there are so many etc.

\subsection{Observables for searches for Supersymmetry}
\label{sub:susy_observables}

standard susy production and decay chains. HT, MET, etc.

\section{Remaining open questions}
\label{sec:theory_remains}

dark energy perhaps, but i actually might ditch this subsection.
