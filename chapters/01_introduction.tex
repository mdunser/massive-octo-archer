% Some commands used in this file
\newcommand{\package}{\emph}

\chapter{Introduction}

The construction of the Large Hadron Collider (LHC) and its experiments at CERN in Geneva over the few last decades has been just
the last step in a long and successful history of particle accelerators that started 
roughly 100 years ago. Just as the first specimens of its kind, the LHC serves -- first 
and foremost -- the purpose of fundamental research. It has been conceived in order to answer
some of the most fundamental questions of modern day physics, such as \textit{`how do particles
acquire mass?'} or \textit{`what is dark matter?'}.
Despite the purely scientific origin of these questions and the improbability of any `practical'
application from any possible answer to them, it is important to note that fundamental
research in general and the research at the LHC in particular do serve a greater and more applicable purpose.

The invention of the world wide web and HTML FIXME(CITE), early developments on touch screens,
research on medical physics with high-power magnet systems as well as medical imaging and the use
of high energy ion beams for tumor treatment are only a few examples of the direct consequences on
daily life which fundamental research on particle physics entails. 

% Apart from this, I regard the education of young scientists such as myself as one of the main accomplishments
% of fundamental science. 
This thesis is dedicated almost exclusively to data analysis of high-energy particle collisions
which took place in the CMS experiment at the LHC. By looking at such collisions, the aforementioned
question about the origin of mass has already been answered FIXME(CITE) with the discovery of the Higgs
boson in 2012 by both the CMS and ATLAS collaborations. The second question, however, remains unanswered
and the work presented in the following is largely devoted to a search for particles which could provide
physicists with a suitable candidate for a dark-matter particle. 

Chapter~\ref{ch:theory} describes the fundamentals of particle physics from a theoretical standpoint,
Chapter FIXME will provide on overview of the CMS experiment. Chapters FIXME to FIXME will then describe
the search for new physics etc. blabla.






